– 2. Escribe un script o conjunto de sentencias DDL que permita crear la base de datos
con todas las tablas replicando el modelo relacional.

create database arg_broker;

use arg_broker;

create table perfil_inversor(
id_perfil_inversor int auto_increment primary key,
nombre varchar(50) not null
);

create table tipos_documentos(
id_tipo_documento int auto_increment primary key,
tipo_documento varchar(20)
);

create table tipos_inversores(
id_tipo_inversor int auto_increment primary key,
tipo_inversor varchar(50) not null
);

create table tipos_transacciones (
id_tipo_transaccion int auto_increment primary key,
nombre_transaccion varchar(50)
);

create table inversores(
id_inversor int auto_increment primary key,
hashed_password varchar(60) not null,
documento varchar(20) unique,
email varchar(50) unique,
telefono varchar(20),
razon_social varchar(50),
id_perfil_inversor int,
id_tipo_documento int,
id_tipo_inversor int,
foreign key (id_perfil_inversor) references perfil_inversor(id_perfil_inversor),
foreign key (id_tipo_documento) references tipos_documentos(id_tipo_documento),
foreign key (id_tipo_inversor) references tipos_inversores(id_tipo_inversor)
);

create table cuentas (
  id_cuenta INT AUTO_INCREMENT PRIMARY KEY,
  numero_cuenta VARCHAR(20) UNIQUE NOT NULL,
  saldo DECIMAL(10, 2) NOT NULL,
  fecha_creacion DATE NOT NULL,
  id_inversor INT,
  FOREIGN KEY (id_inversor) REFERENCES inversores(id_inversor)
);

create table acciones(
id_accion int auto_increment primary key,
simbolo varchar(50) unique,
nombre_empresa varchar(50),
precio_compra DECIMAL(10, 2),
precio_venta DECIMAL(10, 2)
);

-- claves compuestas
create table acciones_por_inversores(
id_inversor int,
id_accion int,
cantidad_acciones int,
precio_compra DECIMAL(10, 2), -- Nuevo campo para el precio de compra
precio_venta DECIMAL(10, 2),  -- Nuevo campo para el precio de venta
primary key (id_inversor, id_accion),
foreign key (id_inversor) references inversores(id_inversor),
foreign key (id_accion) references acciones(id_accion)
);

create table transacciones(
id_transaccion int auto_increment primary key,
cantidad_acciones int,
monto_total decimal(10, 2),
comision decimal(10,2),
fecha_hora datetime,
numero_cuenta varchar(20),
id_tipo_transaccion int,
id_accion int,
foreign key (numero_cuenta) references cuentas(numero_cuenta),
foreign key (id_tipo_transaccion) references tipos_transacciones(id_tipo_transaccion),
foreign key (id_accion) references acciones (id_accion)
);




– 3. Escribe un conjunto de sentencias DML de tipo INSERT que inserten datos iniciales
en la base de datos.

INSERT INTO perfil_inversor values (1, "Pasivo");

INSERT INTO perfil_inversor values (2, "Agresivo");

INSERT INTO tipos_documentos values (1, "CUIL");

INSERT INTO tipos_documentos values (2, "CUIT");

INSERT INTO tipos_documentos values (3, "Pasaporte");

INSERT INTO tipos_inversores values (1, "Persona Fisica");

INSERT INTO tipos_inversores values (2, "Persona Juridica");



INSERT INTO acciones (id_accion, simbolo, nombre_empresa, precio_compra, precio_venta) VALUES 
(1, 'GGAL', 'Grupo Financiero Galicia', 500.00, 550.00),
(2, 'YPFD', 'YPF S.A.', 800.50, 850.00),
(3, 'BMA', 'Banco Macro S.A.', 900.75, 950.00),
(4, 'TXAR', 'Ternium Argentina S.A.', 1200.00, 1250.00),
(5, 'ALUA', 'Aluar Aluminio Argentino S.A.I.C.', 300.25, 320.00),
(6, 'PAMP', 'Pampa Energía S.A.', 1500.10, 1550.00),
(7, 'CRES', 'Cresud S.A.C.I.F. y A.', 600.00, 650.00),
(8, 'COME', 'Sociedad Comercial del Plata S.A.', 180.00, 190.00),
(9, 'CEPU', 'Central Puerto S.A.', 150.50, 160.00),
(10, 'EDN', 'Edenor S.A.', 200.00, 210.00);



INSERT INTO tipos_transacciones (id_tipo_transaccion, nombre_transaccion) VALUES (1, 'Compra'), (2,'Venta');

INSERT INTO `acciones_por_inversores` VALUES (1,1,50,500.00,550.00),
(1,2,50,800.50,850.00),
(2,2,0,800.50,850.00);

INSERT INTO `cuentas` VALUES (1,'RCPCGNH1BYQGNUMNR54Z',927221.25,'2024-10-23',1),
(2,'VI0FWANKSQK7XARBFKGZ',996038.50,'2024-10-25',2);

INSERT INTO `inversores` VALUES (1,'Aldoaldo9*','20207777772','minoldoaldo@gmail.com','35155555555','Aldo',2,1,1),
(2,'Juanjuan1*','20222222222','juan@gmail.com','35162826969','Juan Alvarez',1,1,1);

– 4. Escribe al menos 5 consultas de tipo UPDATE que permitan actualizar datos de los
datos ya insertados.

– Actualizamos saldo de cuenta
UPDATE cuentas
SET saldo = saldo + 1000
WHERE numero_cuenta = 'RCPCGNH1BYQGNUMNR54Z';

– Actualizamos nombre de inversor
UPDATE inversores
SET razon_social = 'Juan Perez'
where documento = '20222222222';

– Actualizamos precio de venta de una accion
UPDATE acciones
SET precio_venta = 900
where simbolo = 'YPFD';

– Actualizamos el telefono de un inversro
UPDATE inversores
SET telefono = 35155555555
WHERE documento = '20207777772';

– Actualizamos nombre de empresa:
UPDATE acciones
set nombre_empresa = 'Nueva YPF S.A'
WHERE simbolo = 'YPFD';

– 5. Escribe al menos 5 consultas de tipo SELECT que permitan obtener datos de los
datos ya insertados.
– Seleccionamos todos los inversores
select * from inversores;

– Seleccionamos el saldo de una cuenta especifica
select saldo from cuentas
where numero_cuenta = 'RCPCGNH1BYQGNUMNR54Z'

– Seleccionamos el precio de acciones mayor a 1000
select * from acciones
where precio_compra > 1000;

– Seleccionamos transacciones en una fecha mayor a 1-1-2024
select * from transacciones
where date(fecha_hora) > '2024-01-01'

– Seleccionamos el nombro y tipo de cada inversor
select razon_social, tipo_inversor
from inversores i
join tipos_inversores t on i.id_tipo_inversor = t.id_tipo_inversor;

– 6. Escribe al menos 3 consultas multitabla que permitan obtener datos de interés para
el caso de estudio.
– Obtener el listado de acciones compradas por cada inversor: 

SELECT i.razon_social, a.simbolo, api.cantidad_acciones FROM acciones_por_inversores api JOIN inversores i ON api.id_inversor = i.id_inversor JOIN acciones a ON api.id_accion = a.id_accion; 

– Detalles de transacciones realizadas, incluyendo el tipo de transacción y la cuenta: 

SELECT t.id_transaccion, t.cantidad_acciones, t.monto_total, tt.nombre_transaccion, c.numero_cuenta FROM transacciones t JOIN tipos_transacciones tt ON t.id_tipo_transaccion = tt.id_tipo_transaccion JOIN cuentas c ON t.numero_cuenta = c.numero_cuenta; 


– Sentencia para calcular rendimiento

SELECT a.simbolo, i.razon_social,
                       SUM(t.monto_total) AS monto_total,
                       SUM(t.comision) AS total_comision,
                       1000000 AS valor_inicial,
                       (1000000 - SUM(t.monto_total) - SUM(t.comision)) AS rendimiento
                FROM transacciones t 
                INNER JOIN cuentas c ON t.numero_cuenta = c.numero_cuenta 
                INNER JOIN acciones a ON t.id_accion = a.id_accion
                INNER JOIN inversores i on i.id_inversor = c.id_inversor
                WHERE c.id_cuenta = 2
                GROUP BY a.simbolo;       